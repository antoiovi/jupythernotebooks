
% Default to the notebook output style

    


% Inherit from the specified cell style.




    
\documentclass[11pt]{article}

    
    
    \usepackage[T1]{fontenc}
    % Nicer default font (+ math font) than Computer Modern for most use cases
    \usepackage{mathpazo}

    % Basic figure setup, for now with no caption control since it's done
    % automatically by Pandoc (which extracts ![](path) syntax from Markdown).
    \usepackage{graphicx}
    % We will generate all images so they have a width \maxwidth. This means
    % that they will get their normal width if they fit onto the page, but
    % are scaled down if they would overflow the margins.
    \makeatletter
    \def\maxwidth{\ifdim\Gin@nat@width>\linewidth\linewidth
    \else\Gin@nat@width\fi}
    \makeatother
    \let\Oldincludegraphics\includegraphics
    % Set max figure width to be 80% of text width, for now hardcoded.
    \renewcommand{\includegraphics}[1]{\Oldincludegraphics[width=.8\maxwidth]{#1}}
    % Ensure that by default, figures have no caption (until we provide a
    % proper Figure object with a Caption API and a way to capture that
    % in the conversion process - todo).
    \usepackage{caption}
    \DeclareCaptionLabelFormat{nolabel}{}
    \captionsetup{labelformat=nolabel}

    \usepackage{adjustbox} % Used to constrain images to a maximum size 
    \usepackage{xcolor} % Allow colors to be defined
    \usepackage{enumerate} % Needed for markdown enumerations to work
    \usepackage{geometry} % Used to adjust the document margins
    \usepackage{amsmath} % Equations
    \usepackage{amssymb} % Equations
    \usepackage{textcomp} % defines textquotesingle
    % Hack from http://tex.stackexchange.com/a/47451/13684:
    \AtBeginDocument{%
        \def\PYZsq{\textquotesingle}% Upright quotes in Pygmentized code
    }
    \usepackage{upquote} % Upright quotes for verbatim code
    \usepackage{eurosym} % defines \euro
    \usepackage[mathletters]{ucs} % Extended unicode (utf-8) support
    \usepackage[utf8x]{inputenc} % Allow utf-8 characters in the tex document
    \usepackage{fancyvrb} % verbatim replacement that allows latex
    \usepackage{grffile} % extends the file name processing of package graphics 
                         % to support a larger range 
    % The hyperref package gives us a pdf with properly built
    % internal navigation ('pdf bookmarks' for the table of contents,
    % internal cross-reference links, web links for URLs, etc.)
    \usepackage{hyperref}
    \usepackage{longtable} % longtable support required by pandoc >1.10
    \usepackage{booktabs}  % table support for pandoc > 1.12.2
    \usepackage[inline]{enumitem} % IRkernel/repr support (it uses the enumerate* environment)
    \usepackage[normalem]{ulem} % ulem is needed to support strikethroughs (\sout)
                                % normalem makes italics be italics, not underlines
    

    
    
    % Colors for the hyperref package
    \definecolor{urlcolor}{rgb}{0,.145,.698}
    \definecolor{linkcolor}{rgb}{.71,0.21,0.01}
    \definecolor{citecolor}{rgb}{.12,.54,.11}

    % ANSI colors
    \definecolor{ansi-black}{HTML}{3E424D}
    \definecolor{ansi-black-intense}{HTML}{282C36}
    \definecolor{ansi-red}{HTML}{E75C58}
    \definecolor{ansi-red-intense}{HTML}{B22B31}
    \definecolor{ansi-green}{HTML}{00A250}
    \definecolor{ansi-green-intense}{HTML}{007427}
    \definecolor{ansi-yellow}{HTML}{DDB62B}
    \definecolor{ansi-yellow-intense}{HTML}{B27D12}
    \definecolor{ansi-blue}{HTML}{208FFB}
    \definecolor{ansi-blue-intense}{HTML}{0065CA}
    \definecolor{ansi-magenta}{HTML}{D160C4}
    \definecolor{ansi-magenta-intense}{HTML}{A03196}
    \definecolor{ansi-cyan}{HTML}{60C6C8}
    \definecolor{ansi-cyan-intense}{HTML}{258F8F}
    \definecolor{ansi-white}{HTML}{C5C1B4}
    \definecolor{ansi-white-intense}{HTML}{A1A6B2}

    % commands and environments needed by pandoc snippets
    % extracted from the output of `pandoc -s`
    \providecommand{\tightlist}{%
      \setlength{\itemsep}{0pt}\setlength{\parskip}{0pt}}
    \DefineVerbatimEnvironment{Highlighting}{Verbatim}{commandchars=\\\{\}}
    % Add ',fontsize=\small' for more characters per line
    \newenvironment{Shaded}{}{}
    \newcommand{\KeywordTok}[1]{\textcolor[rgb]{0.00,0.44,0.13}{\textbf{{#1}}}}
    \newcommand{\DataTypeTok}[1]{\textcolor[rgb]{0.56,0.13,0.00}{{#1}}}
    \newcommand{\DecValTok}[1]{\textcolor[rgb]{0.25,0.63,0.44}{{#1}}}
    \newcommand{\BaseNTok}[1]{\textcolor[rgb]{0.25,0.63,0.44}{{#1}}}
    \newcommand{\FloatTok}[1]{\textcolor[rgb]{0.25,0.63,0.44}{{#1}}}
    \newcommand{\CharTok}[1]{\textcolor[rgb]{0.25,0.44,0.63}{{#1}}}
    \newcommand{\StringTok}[1]{\textcolor[rgb]{0.25,0.44,0.63}{{#1}}}
    \newcommand{\CommentTok}[1]{\textcolor[rgb]{0.38,0.63,0.69}{\textit{{#1}}}}
    \newcommand{\OtherTok}[1]{\textcolor[rgb]{0.00,0.44,0.13}{{#1}}}
    \newcommand{\AlertTok}[1]{\textcolor[rgb]{1.00,0.00,0.00}{\textbf{{#1}}}}
    \newcommand{\FunctionTok}[1]{\textcolor[rgb]{0.02,0.16,0.49}{{#1}}}
    \newcommand{\RegionMarkerTok}[1]{{#1}}
    \newcommand{\ErrorTok}[1]{\textcolor[rgb]{1.00,0.00,0.00}{\textbf{{#1}}}}
    \newcommand{\NormalTok}[1]{{#1}}
    
    % Additional commands for more recent versions of Pandoc
    \newcommand{\ConstantTok}[1]{\textcolor[rgb]{0.53,0.00,0.00}{{#1}}}
    \newcommand{\SpecialCharTok}[1]{\textcolor[rgb]{0.25,0.44,0.63}{{#1}}}
    \newcommand{\VerbatimStringTok}[1]{\textcolor[rgb]{0.25,0.44,0.63}{{#1}}}
    \newcommand{\SpecialStringTok}[1]{\textcolor[rgb]{0.73,0.40,0.53}{{#1}}}
    \newcommand{\ImportTok}[1]{{#1}}
    \newcommand{\DocumentationTok}[1]{\textcolor[rgb]{0.73,0.13,0.13}{\textit{{#1}}}}
    \newcommand{\AnnotationTok}[1]{\textcolor[rgb]{0.38,0.63,0.69}{\textbf{\textit{{#1}}}}}
    \newcommand{\CommentVarTok}[1]{\textcolor[rgb]{0.38,0.63,0.69}{\textbf{\textit{{#1}}}}}
    \newcommand{\VariableTok}[1]{\textcolor[rgb]{0.10,0.09,0.49}{{#1}}}
    \newcommand{\ControlFlowTok}[1]{\textcolor[rgb]{0.00,0.44,0.13}{\textbf{{#1}}}}
    \newcommand{\OperatorTok}[1]{\textcolor[rgb]{0.40,0.40,0.40}{{#1}}}
    \newcommand{\BuiltInTok}[1]{{#1}}
    \newcommand{\ExtensionTok}[1]{{#1}}
    \newcommand{\PreprocessorTok}[1]{\textcolor[rgb]{0.74,0.48,0.00}{{#1}}}
    \newcommand{\AttributeTok}[1]{\textcolor[rgb]{0.49,0.56,0.16}{{#1}}}
    \newcommand{\InformationTok}[1]{\textcolor[rgb]{0.38,0.63,0.69}{\textbf{\textit{{#1}}}}}
    \newcommand{\WarningTok}[1]{\textcolor[rgb]{0.38,0.63,0.69}{\textbf{\textit{{#1}}}}}
    
    
    % Define a nice break command that doesn't care if a line doesn't already
    % exist.
    \def\br{\hspace*{\fill} \\* }
    % Math Jax compatability definitions
    \def\gt{>}
    \def\lt{<}
    % Document parameters
    \title{Aggregazione dati e struttura DataFrameGroupBy}
    
    
    

    % Pygments definitions
    
\makeatletter
\def\PY@reset{\let\PY@it=\relax \let\PY@bf=\relax%
    \let\PY@ul=\relax \let\PY@tc=\relax%
    \let\PY@bc=\relax \let\PY@ff=\relax}
\def\PY@tok#1{\csname PY@tok@#1\endcsname}
\def\PY@toks#1+{\ifx\relax#1\empty\else%
    \PY@tok{#1}\expandafter\PY@toks\fi}
\def\PY@do#1{\PY@bc{\PY@tc{\PY@ul{%
    \PY@it{\PY@bf{\PY@ff{#1}}}}}}}
\def\PY#1#2{\PY@reset\PY@toks#1+\relax+\PY@do{#2}}

\expandafter\def\csname PY@tok@w\endcsname{\def\PY@tc##1{\textcolor[rgb]{0.73,0.73,0.73}{##1}}}
\expandafter\def\csname PY@tok@c\endcsname{\let\PY@it=\textit\def\PY@tc##1{\textcolor[rgb]{0.25,0.50,0.50}{##1}}}
\expandafter\def\csname PY@tok@cp\endcsname{\def\PY@tc##1{\textcolor[rgb]{0.74,0.48,0.00}{##1}}}
\expandafter\def\csname PY@tok@k\endcsname{\let\PY@bf=\textbf\def\PY@tc##1{\textcolor[rgb]{0.00,0.50,0.00}{##1}}}
\expandafter\def\csname PY@tok@kp\endcsname{\def\PY@tc##1{\textcolor[rgb]{0.00,0.50,0.00}{##1}}}
\expandafter\def\csname PY@tok@kt\endcsname{\def\PY@tc##1{\textcolor[rgb]{0.69,0.00,0.25}{##1}}}
\expandafter\def\csname PY@tok@o\endcsname{\def\PY@tc##1{\textcolor[rgb]{0.40,0.40,0.40}{##1}}}
\expandafter\def\csname PY@tok@ow\endcsname{\let\PY@bf=\textbf\def\PY@tc##1{\textcolor[rgb]{0.67,0.13,1.00}{##1}}}
\expandafter\def\csname PY@tok@nb\endcsname{\def\PY@tc##1{\textcolor[rgb]{0.00,0.50,0.00}{##1}}}
\expandafter\def\csname PY@tok@nf\endcsname{\def\PY@tc##1{\textcolor[rgb]{0.00,0.00,1.00}{##1}}}
\expandafter\def\csname PY@tok@nc\endcsname{\let\PY@bf=\textbf\def\PY@tc##1{\textcolor[rgb]{0.00,0.00,1.00}{##1}}}
\expandafter\def\csname PY@tok@nn\endcsname{\let\PY@bf=\textbf\def\PY@tc##1{\textcolor[rgb]{0.00,0.00,1.00}{##1}}}
\expandafter\def\csname PY@tok@ne\endcsname{\let\PY@bf=\textbf\def\PY@tc##1{\textcolor[rgb]{0.82,0.25,0.23}{##1}}}
\expandafter\def\csname PY@tok@nv\endcsname{\def\PY@tc##1{\textcolor[rgb]{0.10,0.09,0.49}{##1}}}
\expandafter\def\csname PY@tok@no\endcsname{\def\PY@tc##1{\textcolor[rgb]{0.53,0.00,0.00}{##1}}}
\expandafter\def\csname PY@tok@nl\endcsname{\def\PY@tc##1{\textcolor[rgb]{0.63,0.63,0.00}{##1}}}
\expandafter\def\csname PY@tok@ni\endcsname{\let\PY@bf=\textbf\def\PY@tc##1{\textcolor[rgb]{0.60,0.60,0.60}{##1}}}
\expandafter\def\csname PY@tok@na\endcsname{\def\PY@tc##1{\textcolor[rgb]{0.49,0.56,0.16}{##1}}}
\expandafter\def\csname PY@tok@nt\endcsname{\let\PY@bf=\textbf\def\PY@tc##1{\textcolor[rgb]{0.00,0.50,0.00}{##1}}}
\expandafter\def\csname PY@tok@nd\endcsname{\def\PY@tc##1{\textcolor[rgb]{0.67,0.13,1.00}{##1}}}
\expandafter\def\csname PY@tok@s\endcsname{\def\PY@tc##1{\textcolor[rgb]{0.73,0.13,0.13}{##1}}}
\expandafter\def\csname PY@tok@sd\endcsname{\let\PY@it=\textit\def\PY@tc##1{\textcolor[rgb]{0.73,0.13,0.13}{##1}}}
\expandafter\def\csname PY@tok@si\endcsname{\let\PY@bf=\textbf\def\PY@tc##1{\textcolor[rgb]{0.73,0.40,0.53}{##1}}}
\expandafter\def\csname PY@tok@se\endcsname{\let\PY@bf=\textbf\def\PY@tc##1{\textcolor[rgb]{0.73,0.40,0.13}{##1}}}
\expandafter\def\csname PY@tok@sr\endcsname{\def\PY@tc##1{\textcolor[rgb]{0.73,0.40,0.53}{##1}}}
\expandafter\def\csname PY@tok@ss\endcsname{\def\PY@tc##1{\textcolor[rgb]{0.10,0.09,0.49}{##1}}}
\expandafter\def\csname PY@tok@sx\endcsname{\def\PY@tc##1{\textcolor[rgb]{0.00,0.50,0.00}{##1}}}
\expandafter\def\csname PY@tok@m\endcsname{\def\PY@tc##1{\textcolor[rgb]{0.40,0.40,0.40}{##1}}}
\expandafter\def\csname PY@tok@gh\endcsname{\let\PY@bf=\textbf\def\PY@tc##1{\textcolor[rgb]{0.00,0.00,0.50}{##1}}}
\expandafter\def\csname PY@tok@gu\endcsname{\let\PY@bf=\textbf\def\PY@tc##1{\textcolor[rgb]{0.50,0.00,0.50}{##1}}}
\expandafter\def\csname PY@tok@gd\endcsname{\def\PY@tc##1{\textcolor[rgb]{0.63,0.00,0.00}{##1}}}
\expandafter\def\csname PY@tok@gi\endcsname{\def\PY@tc##1{\textcolor[rgb]{0.00,0.63,0.00}{##1}}}
\expandafter\def\csname PY@tok@gr\endcsname{\def\PY@tc##1{\textcolor[rgb]{1.00,0.00,0.00}{##1}}}
\expandafter\def\csname PY@tok@ge\endcsname{\let\PY@it=\textit}
\expandafter\def\csname PY@tok@gs\endcsname{\let\PY@bf=\textbf}
\expandafter\def\csname PY@tok@gp\endcsname{\let\PY@bf=\textbf\def\PY@tc##1{\textcolor[rgb]{0.00,0.00,0.50}{##1}}}
\expandafter\def\csname PY@tok@go\endcsname{\def\PY@tc##1{\textcolor[rgb]{0.53,0.53,0.53}{##1}}}
\expandafter\def\csname PY@tok@gt\endcsname{\def\PY@tc##1{\textcolor[rgb]{0.00,0.27,0.87}{##1}}}
\expandafter\def\csname PY@tok@err\endcsname{\def\PY@bc##1{\setlength{\fboxsep}{0pt}\fcolorbox[rgb]{1.00,0.00,0.00}{1,1,1}{\strut ##1}}}
\expandafter\def\csname PY@tok@kc\endcsname{\let\PY@bf=\textbf\def\PY@tc##1{\textcolor[rgb]{0.00,0.50,0.00}{##1}}}
\expandafter\def\csname PY@tok@kd\endcsname{\let\PY@bf=\textbf\def\PY@tc##1{\textcolor[rgb]{0.00,0.50,0.00}{##1}}}
\expandafter\def\csname PY@tok@kn\endcsname{\let\PY@bf=\textbf\def\PY@tc##1{\textcolor[rgb]{0.00,0.50,0.00}{##1}}}
\expandafter\def\csname PY@tok@kr\endcsname{\let\PY@bf=\textbf\def\PY@tc##1{\textcolor[rgb]{0.00,0.50,0.00}{##1}}}
\expandafter\def\csname PY@tok@bp\endcsname{\def\PY@tc##1{\textcolor[rgb]{0.00,0.50,0.00}{##1}}}
\expandafter\def\csname PY@tok@fm\endcsname{\def\PY@tc##1{\textcolor[rgb]{0.00,0.00,1.00}{##1}}}
\expandafter\def\csname PY@tok@vc\endcsname{\def\PY@tc##1{\textcolor[rgb]{0.10,0.09,0.49}{##1}}}
\expandafter\def\csname PY@tok@vg\endcsname{\def\PY@tc##1{\textcolor[rgb]{0.10,0.09,0.49}{##1}}}
\expandafter\def\csname PY@tok@vi\endcsname{\def\PY@tc##1{\textcolor[rgb]{0.10,0.09,0.49}{##1}}}
\expandafter\def\csname PY@tok@vm\endcsname{\def\PY@tc##1{\textcolor[rgb]{0.10,0.09,0.49}{##1}}}
\expandafter\def\csname PY@tok@sa\endcsname{\def\PY@tc##1{\textcolor[rgb]{0.73,0.13,0.13}{##1}}}
\expandafter\def\csname PY@tok@sb\endcsname{\def\PY@tc##1{\textcolor[rgb]{0.73,0.13,0.13}{##1}}}
\expandafter\def\csname PY@tok@sc\endcsname{\def\PY@tc##1{\textcolor[rgb]{0.73,0.13,0.13}{##1}}}
\expandafter\def\csname PY@tok@dl\endcsname{\def\PY@tc##1{\textcolor[rgb]{0.73,0.13,0.13}{##1}}}
\expandafter\def\csname PY@tok@s2\endcsname{\def\PY@tc##1{\textcolor[rgb]{0.73,0.13,0.13}{##1}}}
\expandafter\def\csname PY@tok@sh\endcsname{\def\PY@tc##1{\textcolor[rgb]{0.73,0.13,0.13}{##1}}}
\expandafter\def\csname PY@tok@s1\endcsname{\def\PY@tc##1{\textcolor[rgb]{0.73,0.13,0.13}{##1}}}
\expandafter\def\csname PY@tok@mb\endcsname{\def\PY@tc##1{\textcolor[rgb]{0.40,0.40,0.40}{##1}}}
\expandafter\def\csname PY@tok@mf\endcsname{\def\PY@tc##1{\textcolor[rgb]{0.40,0.40,0.40}{##1}}}
\expandafter\def\csname PY@tok@mh\endcsname{\def\PY@tc##1{\textcolor[rgb]{0.40,0.40,0.40}{##1}}}
\expandafter\def\csname PY@tok@mi\endcsname{\def\PY@tc##1{\textcolor[rgb]{0.40,0.40,0.40}{##1}}}
\expandafter\def\csname PY@tok@il\endcsname{\def\PY@tc##1{\textcolor[rgb]{0.40,0.40,0.40}{##1}}}
\expandafter\def\csname PY@tok@mo\endcsname{\def\PY@tc##1{\textcolor[rgb]{0.40,0.40,0.40}{##1}}}
\expandafter\def\csname PY@tok@ch\endcsname{\let\PY@it=\textit\def\PY@tc##1{\textcolor[rgb]{0.25,0.50,0.50}{##1}}}
\expandafter\def\csname PY@tok@cm\endcsname{\let\PY@it=\textit\def\PY@tc##1{\textcolor[rgb]{0.25,0.50,0.50}{##1}}}
\expandafter\def\csname PY@tok@cpf\endcsname{\let\PY@it=\textit\def\PY@tc##1{\textcolor[rgb]{0.25,0.50,0.50}{##1}}}
\expandafter\def\csname PY@tok@c1\endcsname{\let\PY@it=\textit\def\PY@tc##1{\textcolor[rgb]{0.25,0.50,0.50}{##1}}}
\expandafter\def\csname PY@tok@cs\endcsname{\let\PY@it=\textit\def\PY@tc##1{\textcolor[rgb]{0.25,0.50,0.50}{##1}}}

\def\PYZbs{\char`\\}
\def\PYZus{\char`\_}
\def\PYZob{\char`\{}
\def\PYZcb{\char`\}}
\def\PYZca{\char`\^}
\def\PYZam{\char`\&}
\def\PYZlt{\char`\<}
\def\PYZgt{\char`\>}
\def\PYZsh{\char`\#}
\def\PYZpc{\char`\%}
\def\PYZdl{\char`\$}
\def\PYZhy{\char`\-}
\def\PYZsq{\char`\'}
\def\PYZdq{\char`\"}
\def\PYZti{\char`\~}
% for compatibility with earlier versions
\def\PYZat{@}
\def\PYZlb{[}
\def\PYZrb{]}
\makeatother


    % Exact colors from NB
    \definecolor{incolor}{rgb}{0.0, 0.0, 0.5}
    \definecolor{outcolor}{rgb}{0.545, 0.0, 0.0}



    
    % Prevent overflowing lines due to hard-to-break entities
    \sloppy 
    % Setup hyperref package
    \hypersetup{
      breaklinks=true,  % so long urls are correctly broken across lines
      colorlinks=true,
      urlcolor=urlcolor,
      linkcolor=linkcolor,
      citecolor=citecolor,
      }
    % Slightly bigger margins than the latex defaults
    
    \geometry{verbose,tmargin=1in,bmargin=1in,lmargin=1in,rmargin=1in}
    
    

    \begin{document}
    
    
    \maketitle
    
    

    
    \section{Aggregazione dati e struttura
DataFrameGroupBy}\label{aggregazione-dati-e-struttura-dataframegroupby}

In questo blog vogliamo mostrare la struttura di un DataFrameGroupBy,
ovvero la struttura che si ottiene da un dataframe con operazioni di
aggregazione di dati.

    \subsection{Analisi del problema}\label{analisi-del-problema}

Prima di tutto da dati grezzi sotto forma di file CSV, array,
dictionari, ecc dobbiamo costruire un dataframe :
\includegraphics{attachment:gb01.gif} Poi dal dataframe possiamo
aggregare i dati per esempio per una colonna :
\includegraphics{attachment:gb02.gif} sempre per una colonna :
\includegraphics{attachment:gb03.gif} oppure per due colonne :
\includegraphics{attachment:gb04.gif}

    

    \subsection{Scriviamo un po' di codice}\label{scriviamo-un-po-di-codice}

\paragraph{Importiamo le librerie
necessarie}\label{importiamo-le-librerie-necessarie}

    \begin{Verbatim}[commandchars=\\\{\}]
{\color{incolor}In [{\color{incolor}1}]:} \PY{c+c1}{\PYZsh{} importa le librerie}
        \PY{o}{\PYZpc{}}\PY{k}{matplotlib} inline
        \PY{k+kn}{import} \PY{n+nn}{pandas} \PY{k}{as} \PY{n+nn}{pd}
        \PY{k+kn}{import} \PY{n+nn}{numpy} \PY{k}{as} \PY{n+nn}{np}
\end{Verbatim}


    \subsubsection{Creazione dataframe}\label{creazione-dataframe}

Creiamo un dataframe partendo da un dictionary, a sua volta creato
tramite array. Il dataframe rappresenta delle misure ottenute da a dei
sensori (pressione e temperatura) in diversi periodi, per diverse
citta'.

    \begin{Verbatim}[commandchars=\\\{\}]
{\color{incolor}In [{\color{incolor}2}]:} \PY{n}{localita} \PY{o}{=}         \PY{p}{[}\PY{l+s+s1}{\PYZsq{}}\PY{l+s+s1}{Milano}\PY{l+s+s1}{\PYZsq{}}\PY{p}{,}     \PY{l+s+s1}{\PYZsq{}}\PY{l+s+s1}{Torino}\PY{l+s+s1}{\PYZsq{}}\PY{p}{,}     \PY{l+s+s1}{\PYZsq{}}\PY{l+s+s1}{Milano}\PY{l+s+s1}{\PYZsq{}}\PY{p}{,}     \PY{l+s+s1}{\PYZsq{}}\PY{l+s+s1}{Milano}\PY{l+s+s1}{\PYZsq{}}     \PY{p}{,}\PY{l+s+s1}{\PYZsq{}}\PY{l+s+s1}{Milano}\PY{l+s+s1}{\PYZsq{}}     \PY{p}{,}\PY{l+s+s1}{\PYZsq{}}\PY{l+s+s1}{Milano}\PY{l+s+s1}{\PYZsq{}}     \PY{p}{]}     
        \PY{n}{sensore}\PY{o}{=}           \PY{p}{[}\PY{l+s+s1}{\PYZsq{}}\PY{l+s+s1}{Temperatura}\PY{l+s+s1}{\PYZsq{}}\PY{p}{,}\PY{l+s+s1}{\PYZsq{}}\PY{l+s+s1}{Temperatura}\PY{l+s+s1}{\PYZsq{}}\PY{p}{,}\PY{l+s+s1}{\PYZsq{}}\PY{l+s+s1}{Temperatura}\PY{l+s+s1}{\PYZsq{}}\PY{p}{,}\PY{l+s+s1}{\PYZsq{}}\PY{l+s+s1}{Pressione}\PY{l+s+s1}{\PYZsq{}}  \PY{p}{,}\PY{l+s+s1}{\PYZsq{}}\PY{l+s+s1}{Pressione}\PY{l+s+s1}{\PYZsq{}}  \PY{p}{,}\PY{l+s+s1}{\PYZsq{}}\PY{l+s+s1}{Pressione}\PY{l+s+s1}{\PYZsq{}}  \PY{p}{]}
        \PY{n}{valore}\PY{o}{=}            \PY{p}{[}\PY{l+m+mi}{10}          \PY{p}{,}\PY{l+m+mi}{12}          \PY{p}{,}  \PY{l+m+mi}{14}          \PY{p}{,}  \PY{l+m+mi}{998}         \PY{p}{,}\PY{l+m+mi}{997}          \PY{p}{,}\PY{l+m+mi}{999}          \PY{p}{]}
        \PY{n}{localita}\PY{o}{=}\PY{n}{localita}\PY{o}{+} \PY{p}{[}\PY{l+s+s1}{\PYZsq{}}\PY{l+s+s1}{Torino}\PY{l+s+s1}{\PYZsq{}}\PY{p}{,}     \PY{l+s+s1}{\PYZsq{}}\PY{l+s+s1}{Torino}\PY{l+s+s1}{\PYZsq{}}\PY{p}{,}     \PY{l+s+s1}{\PYZsq{}}\PY{l+s+s1}{Milano}\PY{l+s+s1}{\PYZsq{}}\PY{p}{,}     \PY{l+s+s1}{\PYZsq{}}\PY{l+s+s1}{Milano}\PY{l+s+s1}{\PYZsq{}}     \PY{p}{,}\PY{l+s+s1}{\PYZsq{}}\PY{l+s+s1}{Torino}\PY{l+s+s1}{\PYZsq{}}     \PY{p}{,}\PY{l+s+s1}{\PYZsq{}}\PY{l+s+s1}{Torino}\PY{l+s+s1}{\PYZsq{}}  \PY{p}{]}     
        \PY{n}{sensore}\PY{o}{=}\PY{n}{sensore}\PY{o}{+}   \PY{p}{[}\PY{l+s+s1}{\PYZsq{}}\PY{l+s+s1}{Temperatura}\PY{l+s+s1}{\PYZsq{}}\PY{p}{,}\PY{l+s+s1}{\PYZsq{}}\PY{l+s+s1}{Temperatura}\PY{l+s+s1}{\PYZsq{}}\PY{p}{,}\PY{l+s+s1}{\PYZsq{}}\PY{l+s+s1}{Temperatura}\PY{l+s+s1}{\PYZsq{}}\PY{p}{,}\PY{l+s+s1}{\PYZsq{}}\PY{l+s+s1}{Pressione}\PY{l+s+s1}{\PYZsq{}}  \PY{p}{,}\PY{l+s+s1}{\PYZsq{}}\PY{l+s+s1}{Pressione}\PY{l+s+s1}{\PYZsq{}}  \PY{p}{,}\PY{l+s+s1}{\PYZsq{}}\PY{l+s+s1}{Pressione}\PY{l+s+s1}{\PYZsq{}}  \PY{p}{]}
        \PY{n}{valore}\PY{o}{=}\PY{n}{valore}\PY{o}{+}     \PY{p}{[}\PY{l+m+mi}{10}          \PY{p}{,}\PY{l+m+mi}{12}          \PY{p}{,}  \PY{l+m+mi}{14}          \PY{p}{,}  \PY{l+m+mi}{998}         \PY{p}{,}\PY{l+m+mi}{997}         \PY{p}{,}\PY{l+m+mi}{999}          \PY{p}{]}
        
        \PY{n}{tabella}\PY{o}{=}\PY{p}{\PYZob{}}\PY{l+s+s1}{\PYZsq{}}\PY{l+s+s1}{localita}\PY{l+s+s1}{\PYZsq{}}\PY{p}{:}\PY{n}{localita}\PY{p}{,}\PY{l+s+s1}{\PYZsq{}}\PY{l+s+s1}{sensore}\PY{l+s+s1}{\PYZsq{}}\PY{p}{:}\PY{n}{sensore}\PY{p}{,}\PY{l+s+s1}{\PYZsq{}}\PY{l+s+s1}{valore}\PY{l+s+s1}{\PYZsq{}}\PY{p}{:}\PY{n}{valore}\PY{p}{\PYZcb{}}
        \PY{n}{df}\PY{o}{=}\PY{n}{pd}\PY{o}{.}\PY{n}{DataFrame}\PY{p}{(}\PY{n}{tabella}\PY{p}{)}
\end{Verbatim}


\begin{Verbatim}[commandchars=\\\{\}]
{\color{outcolor}Out[{\color{outcolor}2}]:}    localita      sensore  valore
        0    Milano  Temperatura      10
        1    Torino  Temperatura      12
        2    Milano  Temperatura      14
        3    Milano    Pressione     998
        4    Milano    Pressione     997
        5    Milano    Pressione     999
        6    Torino  Temperatura      10
        7    Torino  Temperatura      12
        8    Milano  Temperatura      14
        9    Milano    Pressione     998
        10   Torino    Pressione     997
        11   Torino    Pressione     999
\end{Verbatim}
            
    \section{Aggregazione dati}\label{aggregazione-dati}

    \subsubsection{Group by su UNA colonna}\label{group-by-su-una-colonna}

Proviamo a fare un groupby su una colonna , per esempio la colonna
località. Le elaborazioni vengono effettuate sui dati corrispondenti
alla localita'.

Quindi come count ho il numero totale degli elemnti per ogni località,
mentre come media dei valori numerici ho dei valori non attendibili in
qunato ho la media delle pressioni e temperature insieme.

Il tipo dati ottenuto `e 'pandas.core.groupby.groupby.DataFrameGroupBy'

    \begin{Verbatim}[commandchars=\\\{\}]
{\color{incolor}In [{\color{incolor}9}]:} \PY{n}{gb}\PY{o}{=}\PY{n}{df}\PY{o}{.}\PY{n}{groupby}\PY{p}{(}\PY{l+s+s1}{\PYZsq{}}\PY{l+s+s1}{localita}\PY{l+s+s1}{\PYZsq{}}\PY{p}{)}
        \PY{n+nb}{print}\PY{p}{(}\PY{l+s+s2}{\PYZdq{}}\PY{l+s+s2}{Tipo gb :}\PY{l+s+s2}{\PYZdq{}}\PY{p}{,}\PY{n+nb}{type}\PY{p}{(}\PY{n}{gb}\PY{p}{)}\PY{p}{)}
        \PY{n}{gb}\PY{o}{.}\PY{n}{describe}\PY{p}{(}\PY{p}{)}
\end{Verbatim}


    \begin{Verbatim}[commandchars=\\\{\}]
Tipo gb : <class 'pandas.core.groupby.groupby.DataFrameGroupBy'>

    \end{Verbatim}

\begin{Verbatim}[commandchars=\\\{\}]
{\color{outcolor}Out[{\color{outcolor}9}]:}          valore                                                         
                  count        mean         std   min   25\%    50\%    75\%    max
        localita                                                                
        Milano      7.0  575.714286  526.684825  10.0  14.0  997.0  998.0  999.0
        Torino      5.0  406.000000  540.420669  10.0  12.0   12.0  997.0  999.0
\end{Verbatim}
            
    Calcoliamo medie e conteggi . Se si fanno calcoli numerici su campi non
nuemrici si ha un errore (sensore : campo NON numerico, no si puo
calcolare la media print(gb.sensore.mean()))

    \begin{Verbatim}[commandchars=\\\{\}]
{\color{incolor}In [{\color{incolor}13}]:} \PY{n+nb}{print}\PY{p}{(}\PY{n}{gb}\PY{o}{.}\PY{n}{sensore}\PY{p}{)}
         \PY{n+nb}{print}\PY{p}{(}\PY{n}{gb}\PY{o}{.}\PY{n}{sensore}\PY{o}{.}\PY{n}{count}\PY{p}{(}\PY{p}{)}\PY{p}{)}
         \PY{n+nb}{print}\PY{p}{(}\PY{n}{gb}\PY{o}{.}\PY{n}{valore}\PY{o}{.}\PY{n}{mean}\PY{p}{(}\PY{p}{)}\PY{p}{)}
         \PY{c+c1}{\PYZsh{} sensore : campo NON numerico, no si puo calcolare la media print(gb.sensore.mean())}
\end{Verbatim}


    \begin{Verbatim}[commandchars=\\\{\}]
<pandas.core.groupby.groupby.SeriesGroupBy object at 0x7f940b46fc18>
localita
Milano    7
Torino    5
Name: sensore, dtype: int64
localita
Milano    575.714286
Torino    406.000000
Name: valore, dtype: float64

    \end{Verbatim}

    \subsubsection{groupby su DUE colonne}\label{groupby-su-due-colonne}

    \begin{Verbatim}[commandchars=\\\{\}]
{\color{incolor}In [{\color{incolor}18}]:} \PY{n}{gb}\PY{p}{[}\PY{l+s+s1}{\PYZsq{}}\PY{l+s+s1}{valore}\PY{l+s+s1}{\PYZsq{}}\PY{p}{]}
\end{Verbatim}


\begin{Verbatim}[commandchars=\\\{\}]
{\color{outcolor}Out[{\color{outcolor}18}]:} <pandas.core.groupby.groupby.SeriesGroupBy object at 0x7f940b47d710>
\end{Verbatim}
            
    Effettuiamo un groupby su due colonne (località e sensore). Prima di
tutto notiamo che i campi vanno passati come array( tra parentesi
quadre). I dati sono aggregati prima per citta e poi per semsore.

    \begin{Verbatim}[commandchars=\\\{\}]
{\color{incolor}In [{\color{incolor}19}]:} \PY{n}{gb2}\PY{o}{=}\PY{n}{df}\PY{o}{.}\PY{n}{groupby}\PY{p}{(}\PY{p}{[}\PY{l+s+s1}{\PYZsq{}}\PY{l+s+s1}{localita}\PY{l+s+s1}{\PYZsq{}}\PY{p}{,}\PY{l+s+s1}{\PYZsq{}}\PY{l+s+s1}{sensore}\PY{l+s+s1}{\PYZsq{}}\PY{p}{]}\PY{p}{)}
         \PY{n}{gb2}\PY{o}{.}\PY{n}{describe}\PY{p}{(}\PY{p}{)}
\end{Verbatim}


\begin{Verbatim}[commandchars=\\\{\}]
{\color{outcolor}Out[{\color{outcolor}19}]:}                      valore                                              \textbackslash{}
                               count        mean       std    min     25\%    50\%   
         localita sensore                                                          
         Milano   Pressione      4.0  998.000000  0.816497  997.0  997.75  998.0   
                  Temperatura    3.0   12.666667  2.309401   10.0   12.00   14.0   
         Torino   Pressione      2.0  998.000000  1.414214  997.0  997.50  998.0   
                  Temperatura    3.0   11.333333  1.154701   10.0   11.00   12.0   
         
                                              
                                  75\%    max  
         localita sensore                     
         Milano   Pressione    998.25  999.0  
                  Temperatura   14.00   14.0  
         Torino   Pressione    998.50  999.0  
                  Temperatura   12.00   12.0  
\end{Verbatim}
            
    Possiamo quindi avere dei dati per localita e sensore :

    \begin{Verbatim}[commandchars=\\\{\}]
{\color{incolor}In [{\color{incolor}20}]:} \PY{n+nb}{print}\PY{p}{(}\PY{n}{gb2}\PY{o}{.}\PY{n}{valore}\PY{o}{.}\PY{n}{mean}\PY{p}{(}\PY{p}{)}\PY{p}{)}
         \PY{n+nb}{print}\PY{p}{(}\PY{n}{gb2}\PY{o}{.}\PY{n}{sensore}\PY{o}{.}\PY{n}{count}\PY{p}{(}\PY{p}{)}\PY{p}{)}
\end{Verbatim}


    \begin{Verbatim}[commandchars=\\\{\}]
localita  sensore    
Milano    Pressione      998.000000
          Temperatura     12.666667
Torino    Pressione      998.000000
          Temperatura     11.333333
Name: valore, dtype: float64
localita  sensore    
Milano    Pressione      4
          Temperatura    3
Torino    Pressione      2
          Temperatura    3
Name: sensore, dtype: int64

    \end{Verbatim}

    \subsection{Tipi dati}\label{tipi-dati}

    \paragraph{DataFrameGroupBy}\label{dataframegroupby}

L'operazione groupy(...) restituisce un tipo di dato
'pandas.core.groupby.groupby.DataFrameGroupBy' \#\#\#\# Series Il tipo
di estrazione ottenuta effettuando calcoli su un campo è una
pandas.core.Series .

    \begin{Verbatim}[commandchars=\\\{\}]
{\color{incolor}In [{\color{incolor}19}]:} \PY{n+nb}{print}\PY{p}{(}\PY{n+nb}{type}\PY{p}{(}\PY{n}{gb2}\PY{p}{)}\PY{p}{)}
         \PY{n+nb}{print}\PY{p}{(}\PY{n+nb}{type}\PY{p}{(}\PY{n}{gb2}\PY{o}{.}\PY{n}{valore}\PY{o}{.}\PY{n}{mean}\PY{p}{(}\PY{p}{)}\PY{p}{)}\PY{p}{)}
\end{Verbatim}


    \begin{Verbatim}[commandchars=\\\{\}]
<class 'pandas.core.groupby.groupby.DataFrameGroupBy'>
<class 'pandas.core.series.Series'>

    \end{Verbatim}

    \subparagraph{In sintesi :}\label{in-sintesi}

    \begin{figure}
\centering
\includegraphics{attachment:groupby_2f.gif}
\caption{groupby\_2f.gif}
\end{figure}

    \subsection{UNSTACK}\label{unstack}

Con l'opzione unstack posso spostare una colonna aggregata sulla
intestazione. Questa operazione mi trasforma la Series in dataframe, e
posso in tale maniera effettuare operazoni specifiche del dataframe .
Quindi nel codice sotto eseguimo questa trasformazione :

da groupby -\/-\textgreater{} valore.mean()-\/-\textgreater{} Serie
-\/-\/-\textgreater{} unstack(..) -\/-\textgreater{} dataframe :

    \begin{Verbatim}[commandchars=\\\{\}]
{\color{incolor}In [{\color{incolor}21}]:} \PY{c+c1}{\PYZsh{} Porto la colonna 0 (localita) sull \PYZsq{}header }
         \PY{n}{df0}\PY{o}{=}\PY{n}{gb2}\PY{o}{.}\PY{n}{valore}\PY{o}{.}\PY{n}{mean}\PY{p}{(}\PY{p}{)}\PY{o}{.}\PY{n}{unstack}\PY{p}{(}\PY{l+m+mi}{0}\PY{p}{)}
         \PY{c+c1}{\PYZsh{} Porto la colonna 1 (sensore) sull \PYZsq{}header }
         \PY{n}{df1}\PY{o}{=}\PY{n}{gb2}\PY{o}{.}\PY{n}{valore}\PY{o}{.}\PY{n}{mean}\PY{p}{(}\PY{p}{)}\PY{o}{.}\PY{n}{unstack}\PY{p}{(}\PY{l+m+mi}{1}\PY{p}{)}
         \PY{n+nb}{print}\PY{p}{(}\PY{l+s+s2}{\PYZdq{}}\PY{l+s+s2}{unstack(0) :Porto la colonna 0 (localita) sull }\PY{l+s+s2}{\PYZsq{}}\PY{l+s+s2}{header}\PY{l+s+se}{\PYZbs{}n}\PY{l+s+s2}{\PYZdq{}}\PY{p}{)}
         \PY{n+nb}{print}\PY{p}{(}\PY{n}{df0}\PY{p}{)}
         \PY{n+nb}{print}\PY{p}{(}\PY{l+s+s2}{\PYZdq{}}\PY{l+s+se}{\PYZbs{}n}\PY{l+s+s2}{unstack(1) :Porto la colonna 1 (sensore) sull }\PY{l+s+s2}{\PYZsq{}}\PY{l+s+s2}{header}\PY{l+s+se}{\PYZbs{}n}\PY{l+s+s2}{\PYZdq{}}\PY{p}{)}
         \PY{n+nb}{print}\PY{p}{(}\PY{n}{df1}\PY{p}{)}
\end{Verbatim}


    \begin{Verbatim}[commandchars=\\\{\}]
unstack(0) :Porto la colonna 0 (localita) sull 'header

localita         Milano      Torino
sensore                            
Pressione    998.000000  998.000000
Temperatura   12.666667   11.333333

unstack(1) :Porto la colonna 1 (sensore) sull 'header

sensore   Pressione  Temperatura
localita                        
Milano        998.0    12.666667
Torino        998.0    11.333333

    \end{Verbatim}

    \begin{Verbatim}[commandchars=\\\{\}]
{\color{incolor}In [{\color{incolor}22}]:} \PY{n}{gb2}\PY{o}{.}\PY{n}{valore}\PY{o}{.}\PY{n}{mean}\PY{p}{(}\PY{p}{)}\PY{o}{.}\PY{n}{unstack}\PY{p}{(}\PY{l+s+s1}{\PYZsq{}}\PY{l+s+s1}{sensore}\PY{l+s+s1}{\PYZsq{}}\PY{p}{)}
\end{Verbatim}


\begin{Verbatim}[commandchars=\\\{\}]
{\color{outcolor}Out[{\color{outcolor}22}]:} sensore   Pressione  Temperatura
         localita                        
         Milano        998.0    12.666667
         Torino        998.0    11.333333
\end{Verbatim}
            
    \begin{figure}
\centering
\includegraphics{attachment:groupby_unstack.gif}
\caption{groupby\_unstack.gif}
\end{figure}

    Possiamo fare operazioni sui dati trasformati in dataframe :

    \begin{Verbatim}[commandchars=\\\{\}]
{\color{incolor}In [{\color{incolor}37}]:} \PY{n+nb}{print}\PY{p}{(}\PY{n}{df0}\PY{o}{.}\PY{n}{iloc}\PY{p}{[}\PY{l+m+mi}{0}\PY{p}{]}\PY{p}{[}\PY{l+m+mi}{1}\PY{p}{]}\PY{p}{)}
         \PY{n+nb}{print}\PY{p}{(}\PY{n}{df1}\PY{o}{.}\PY{n}{iloc}\PY{p}{[}\PY{l+m+mi}{0}\PY{p}{]}\PY{p}{[}\PY{l+m+mi}{1}\PY{p}{]}\PY{p}{)}
\end{Verbatim}


    \begin{Verbatim}[commandchars=\\\{\}]
998.0
12.666666666666666

    \end{Verbatim}

    \section{Iterazioni su
DataFrameGroupBy}\label{iterazioni-su-dataframegroupby}

    \subsection{DataFrameGroupBy ottenuta con una
colonna}\label{dataframegroupby-ottenuta-con-una-colonna}

\subsubsection{Iterazione ad Un parametro (for x in
gb)}\label{iterazione-ad-un-parametro-for-x-in-gb}

Ogni iterazione restitusce come elemento una TUPLA (variablie gruppo
nell'esempio ) Il primo elemento della tupla restituisce il nome del
gruppo (gruppo{[}0{]}) Il secondo elemento della tupla (gruppo{[}1{]})
restituisce un dataframe che rappresenta i dati .

    \begin{Verbatim}[commandchars=\\\{\}]
{\color{incolor}In [{\color{incolor}83}]:} \PY{c+c1}{\PYZsh{} i : per eseguire il ciclo solo sul primo elemento}
         \PY{n}{i}\PY{o}{=}\PY{l+m+mi}{0}
         \PY{k}{for} \PY{n}{gruppo} \PY{o+ow}{in} \PY{n}{gb}\PY{p}{:}
             \PY{n}{i}\PY{o}{=}\PY{n}{i}\PY{o}{+}\PY{l+m+mi}{1}
             \PY{n+nb}{print}\PY{p}{(}\PY{n}{gruppo}\PY{p}{)}
             \PY{n+nb}{print}\PY{p}{(}\PY{l+s+s2}{\PYZdq{}}\PY{l+s+s2}{Type gruppo : }\PY{l+s+s2}{\PYZdq{}}\PY{p}{,}\PY{n+nb}{type}\PY{p}{(}\PY{n}{gruppo}\PY{p}{)}\PY{p}{)}
             \PY{n+nb}{print}\PY{p}{(}\PY{l+s+s2}{\PYZdq{}}\PY{l+s+s2}{Type gruppo[0] : }\PY{l+s+s2}{\PYZdq{}}\PY{p}{,}\PY{n+nb}{type}\PY{p}{(}\PY{n}{gruppo}\PY{p}{[}\PY{l+m+mi}{0}\PY{p}{]}\PY{p}{)}\PY{p}{)}
             \PY{n+nb}{print}\PY{p}{(}\PY{l+s+s2}{\PYZdq{}}\PY{l+s+s2}{Type gruppo[1] : }\PY{l+s+s2}{\PYZdq{}}\PY{p}{,}\PY{n+nb}{type}\PY{p}{(}\PY{n}{gruppo}\PY{p}{[}\PY{l+m+mi}{1}\PY{p}{]}\PY{p}{)}\PY{p}{)}
             \PY{c+c1}{\PYZsh{}print(type(x))}
             \PY{c+c1}{\PYZsh{}print(\PYZdq{}X[0] =\PYZdq{},x[0],\PYZdq{}X[1]=\PYZdq{},x[1],type(x[1]))}
             \PY{k}{if} \PY{n}{i} \PY{o}{\PYZgt{}} \PY{l+m+mi}{0}\PY{p}{:}
                 \PY{k}{break}
\end{Verbatim}


    \begin{Verbatim}[commandchars=\\\{\}]
('Milano',   localita      sensore  valore
0   Milano  Temperatura      10
2   Milano  Temperatura      14
3   Milano    Pressione     998
4   Milano    Pressione     997
5   Milano    Pressione     999
8   Milano  Temperatura      14
9   Milano    Pressione     998)
<class 'tuple'>
<class 'str'>
<class 'pandas.core.frame.DataFrame'>

    \end{Verbatim}

    \subsubsection{Iterazione a Due parametri (for x,y in
gb)}\label{iterazione-a-due-parametri-for-xy-in-gb}

Con due parametri viene già scomposta la tupla mostrata in precedenza.

Sul primo elemento viene caricato ad agni ciclo il nome del gruppo come
stringa (il campo aggregato); In questo esempio il nome della citta.

Sul secondo elemento (Y) vengono caricati i dati sotto forma di
dataframe con il campo aggregato filtrato .

    \begin{Verbatim}[commandchars=\\\{\}]
{\color{incolor}In [{\color{incolor}88}]:} \PY{c+c1}{\PYZsh{} i : per eseguire il ciclo solo sul primo elemento}
         \PY{n}{i}\PY{o}{=}\PY{l+m+mi}{0}
         \PY{k}{for} \PY{n}{nomegruppo}\PY{p}{,}\PY{n}{dfgruppo} \PY{o+ow}{in} \PY{n}{gb}\PY{p}{:}
             \PY{n}{i}\PY{o}{=}\PY{n}{i}\PY{o}{+}\PY{l+m+mi}{1}
             \PY{n+nb}{print}\PY{p}{(}\PY{l+s+s2}{\PYZdq{}}\PY{l+s+s2}{nomegruppo : }\PY{l+s+s2}{\PYZdq{}}\PY{p}{,}\PY{n}{nomegruppo}\PY{p}{)}
             \PY{n+nb}{print}\PY{p}{(}\PY{l+s+s2}{\PYZdq{}}\PY{l+s+s2}{Tipo nome gruppo :}\PY{l+s+s2}{\PYZdq{}}\PY{p}{,}\PY{n+nb}{type}\PY{p}{(}\PY{n}{nomegruppo}\PY{p}{)}\PY{p}{)}
             \PY{n+nb}{print}\PY{p}{(}\PY{l+s+s2}{\PYZdq{}}\PY{l+s+se}{\PYZbs{}n}\PY{l+s+s2}{dfgruppo : }\PY{l+s+s2}{\PYZdq{}}\PY{p}{)}
             \PY{n+nb}{print}\PY{p}{(}\PY{n}{dfgruppo}\PY{p}{)}
             \PY{n+nb}{print}\PY{p}{(}\PY{l+s+s2}{\PYZdq{}}\PY{l+s+s2}{Tipo dfgruppo :}\PY{l+s+s2}{\PYZdq{}}\PY{p}{,}\PY{n+nb}{type}\PY{p}{(}\PY{n}{dfgruppo}\PY{p}{)}\PY{p}{)}
             \PY{k}{if} \PY{n}{i}\PY{o}{\PYZgt{}}\PY{l+m+mi}{0} \PY{p}{:}
                 \PY{k}{break}
\end{Verbatim}


    \begin{Verbatim}[commandchars=\\\{\}]
nomegruppo :  Milano
Tipo nome gruppo : <class 'str'>

dfgruppo : 
  localita      sensore  valore
0   Milano  Temperatura      10
2   Milano  Temperatura      14
3   Milano    Pressione     998
4   Milano    Pressione     997
5   Milano    Pressione     999
8   Milano  Temperatura      14
9   Milano    Pressione     998
Tipo dfgruppo : <class 'pandas.core.frame.DataFrame'>

    \end{Verbatim}

    \subsection{DataFrameGroupBy creato da piu'
aggregazioni}\label{dataframegroupby-creato-da-piu-aggregazioni}

Utilizzando il DataFrameGroupBy ottenuto aggregando più colonne , come
mome del gruppo, nella iterazione, si ottiene una tupla con i nomi dei
campi, mentre nella seconda variabile della iterazione si ottiene sempre
un dataframe :

    \begin{Verbatim}[commandchars=\\\{\}]
{\color{incolor}In [{\color{incolor}89}]:} \PY{c+c1}{\PYZsh{} i : per eseguire il ciclo solo sul primo elemento}
         \PY{n}{i}\PY{o}{=}\PY{l+m+mi}{0}
         \PY{k}{for} \PY{n}{nomegruppo}\PY{p}{,}\PY{n}{dfgruppo} \PY{o+ow}{in} \PY{n}{gb2}\PY{p}{:}
             \PY{n}{i}\PY{o}{=}\PY{n}{i}\PY{o}{+}\PY{l+m+mi}{1}
             \PY{n+nb}{print}\PY{p}{(}\PY{l+s+s2}{\PYZdq{}}\PY{l+s+s2}{nomegruppo : }\PY{l+s+s2}{\PYZdq{}}\PY{p}{,}\PY{n}{nomegruppo}\PY{p}{)}
             \PY{n+nb}{print}\PY{p}{(}\PY{l+s+s2}{\PYZdq{}}\PY{l+s+s2}{Tipo nome gruppo :}\PY{l+s+s2}{\PYZdq{}}\PY{p}{,}\PY{n+nb}{type}\PY{p}{(}\PY{n}{nomegruppo}\PY{p}{)}\PY{p}{)}
             \PY{n+nb}{print}\PY{p}{(}\PY{l+s+s2}{\PYZdq{}}\PY{l+s+se}{\PYZbs{}n}\PY{l+s+s2}{dfgruppo : }\PY{l+s+s2}{\PYZdq{}}\PY{p}{)}
             \PY{n+nb}{print}\PY{p}{(}\PY{n}{dfgruppo}\PY{p}{)}
             \PY{n+nb}{print}\PY{p}{(}\PY{l+s+s2}{\PYZdq{}}\PY{l+s+s2}{Tipo dfgruppo :}\PY{l+s+s2}{\PYZdq{}}\PY{p}{,}\PY{n+nb}{type}\PY{p}{(}\PY{n}{dfgruppo}\PY{p}{)}\PY{p}{)}
             \PY{k}{if} \PY{n}{i}\PY{o}{\PYZgt{}}\PY{l+m+mi}{0} \PY{p}{:}
                 \PY{k}{break}
\end{Verbatim}


    \begin{Verbatim}[commandchars=\\\{\}]
nomegruppo :  ('Milano', 'Pressione')
Tipo nome gruppo : <class 'tuple'>

dfgruppo : 
  localita    sensore  valore
3   Milano  Pressione     998
4   Milano  Pressione     997
5   Milano  Pressione     999
9   Milano  Pressione     998
Tipo dfgruppo : <class 'pandas.core.frame.DataFrame'>

    \end{Verbatim}

    \section{Operazioni varie per estrarre dati da un
DataFrameGroupBy}\label{operazioni-varie-per-estrarre-dati-da-un-dataframegroupby}

    \begin{Verbatim}[commandchars=\\\{\}]
{\color{incolor}In [{\color{incolor} }]:} \PY{n}{Otteniamo} \PY{n}{i} \PY{n}{nomi} \PY{n}{dei} \PY{n}{gruppi}\PY{p}{,} \PY{n}{sotto} \PY{n}{forma} \PY{n}{di} \PY{n}{dict\PYZus{}key}
\end{Verbatim}


    \begin{Verbatim}[commandchars=\\\{\}]
{\color{incolor}In [{\color{incolor}115}]:} \PY{n}{dk}\PY{o}{=}\PY{n}{gb2}\PY{o}{.}\PY{n}{groups}\PY{o}{.}\PY{n}{keys}\PY{p}{(}\PY{p}{)}
          \PY{n+nb}{print}\PY{p}{(}\PY{n}{dk}\PY{p}{)}
          \PY{n}{lstdk}\PY{o}{=}\PY{n+nb}{list}\PY{p}{(}\PY{n}{dk}\PY{p}{)}
          \PY{n}{lstdk}\PY{p}{[}\PY{l+m+mi}{2}\PY{p}{]}
\end{Verbatim}


    \begin{Verbatim}[commandchars=\\\{\}]
dict\_keys([('Milano', 'Pressione'), ('Milano', 'Temperatura'), ('Torino', 'Pressione'), ('Torino', 'Temperatura')])

    \end{Verbatim}

\begin{Verbatim}[commandchars=\\\{\}]
{\color{outcolor}Out[{\color{outcolor}115}]:} ('Torino', 'Pressione')
\end{Verbatim}
            
    Otteniamo gli indici dei dati di ogni gruppo ( tipo dict\_items) e
cerchiamo i valori nel dataframe :

    \begin{Verbatim}[commandchars=\\\{\}]
{\color{incolor}In [{\color{incolor}141}]:} \PY{n}{ditm}\PY{o}{=}\PY{n}{gb2}\PY{o}{.}\PY{n}{groups}\PY{o}{.}\PY{n}{items}\PY{p}{(}\PY{p}{)}
          \PY{n+nb}{print}\PY{p}{(}\PY{l+s+s2}{\PYZdq{}}\PY{l+s+s2}{gb2.groups.items() (tipo dict\PYZus{}item): }\PY{l+s+se}{\PYZbs{}n}\PY{l+s+s2}{\PYZdq{}}\PY{p}{,}\PY{n}{ditm}\PY{p}{)}
          \PY{n}{items}\PY{o}{=}\PY{n+nb}{list}\PY{p}{(}\PY{n}{ditm}\PY{p}{)}
          \PY{n}{gr}\PY{o}{=}\PY{l+m+mi}{0}
          \PY{n+nb}{print}\PY{p}{(}\PY{l+s+s2}{\PYZdq{}}\PY{l+s+se}{\PYZbs{}n}\PY{l+s+s2}{ items=list(ditm), elenco dei gruppi. }\PY{l+s+s2}{\PYZdq{}}\PY{p}{)}
          \PY{n+nb}{print}\PY{p}{(}\PY{l+s+s2}{\PYZdq{}}\PY{l+s+se}{\PYZbs{}n}\PY{l+s+s2}{ items[}\PY{l+s+si}{\PYZob{}\PYZcb{}}\PY{l+s+s2}{] (tupla) , rappresenta il gruppo [}\PY{l+s+si}{\PYZob{}\PYZcb{}}\PY{l+s+s2}{]:  }\PY{l+s+se}{\PYZbs{}n}\PY{l+s+s2}{\PYZdq{}}\PY{o}{.}\PY{n}{format}\PY{p}{(}\PY{n}{gr}\PY{p}{,}\PY{n}{gr}\PY{p}{)}\PY{p}{,}\PY{n}{items}\PY{p}{[}\PY{l+m+mi}{0}\PY{p}{]}\PY{p}{)}
          
          \PY{n+nb}{print}\PY{p}{(}\PY{l+s+s2}{\PYZdq{}}\PY{l+s+se}{\PYZbs{}n}\PY{l+s+s2}{ Nome   del gruppo }\PY{l+s+si}{\PYZob{}\PYZcb{}}\PY{l+s+s2}{ \PYZhy{}\PYZhy{}\PYZgt{} items[}\PY{l+s+si}{\PYZob{}\PYZcb{}}\PY{l+s+s2}{][0] : }\PY{l+s+se}{\PYZbs{}n}\PY{l+s+s2}{\PYZdq{}}\PY{o}{.}\PY{n}{format}\PY{p}{(}\PY{n}{gr}\PY{p}{,}\PY{n}{gr}\PY{p}{)}\PY{p}{,}\PY{n}{items}\PY{p}{[}\PY{n}{gr}\PY{p}{]}\PY{p}{[}\PY{l+m+mi}{0}\PY{p}{]}\PY{p}{)}
          \PY{n+nb}{print}\PY{p}{(}\PY{l+s+s2}{\PYZdq{}}\PY{l+s+se}{\PYZbs{}n}\PY{l+s+s2}{ Indici del gruppo }\PY{l+s+si}{\PYZob{}\PYZcb{}}\PY{l+s+s2}{ \PYZhy{}\PYZhy{}\PYZgt{} items[}\PY{l+s+si}{\PYZob{}\PYZcb{}}\PY{l+s+s2}{][1] : }\PY{l+s+se}{\PYZbs{}n}\PY{l+s+s2}{\PYZdq{}}\PY{o}{.}\PY{n}{format}\PY{p}{(}\PY{n}{gr}\PY{p}{,}\PY{n}{gr}\PY{p}{)}\PY{p}{,}\PY{n}{items}\PY{p}{[}\PY{n}{gr}\PY{p}{]}\PY{p}{[}\PY{l+m+mi}{1}\PY{p}{]}\PY{p}{)}
          \PY{n+nb}{print}\PY{p}{(}\PY{l+s+s2}{\PYZdq{}}\PY{l+s+se}{\PYZbs{}n}\PY{l+s+s2}{Ottengo le righe dal dataframe in base agli indici del gruppo :}\PY{l+s+s2}{\PYZdq{}}\PY{p}{)}
          \PY{k}{for} \PY{n}{i} \PY{o+ow}{in} \PY{n}{items}\PY{p}{[}\PY{l+m+mi}{0}\PY{p}{]}\PY{p}{[}\PY{l+m+mi}{1}\PY{p}{]}\PY{p}{:}
              \PY{n+nb}{print}\PY{p}{(}\PY{l+s+s2}{\PYZdq{}}\PY{l+s+se}{\PYZbs{}n}\PY{l+s+s2}{ \PYZhy{}\PYZhy{}\PYZhy{}\PYZhy{}\PYZhy{} Riga del dataframe }\PY{l+s+si}{\PYZob{}\PYZcb{}}\PY{l+s+s2}{\PYZdq{}}\PY{o}{.}\PY{n}{format}\PY{p}{(}\PY{n}{i}\PY{p}{)}\PY{p}{)}
              \PY{n+nb}{print}\PY{p}{(}\PY{n}{df}\PY{o}{.}\PY{n}{iloc}\PY{p}{[}\PY{n}{i}\PY{p}{]}\PY{p}{)}
\end{Verbatim}


    \begin{Verbatim}[commandchars=\\\{\}]
gb2.groups.items() (tipo dict\_item): 
 dict\_items([(('Milano', 'Pressione'), Int64Index([3, 4, 5, 9], dtype='int64')), (('Milano', 'Temperatura'), Int64Index([0, 2, 8], dtype='int64')), (('Torino', 'Pressione'), Int64Index([10, 11], dtype='int64')), (('Torino', 'Temperatura'), Int64Index([1, 6, 7], dtype='int64'))])

 items=list(ditm), elenco dei gruppi. 

 items[0] (tupla) , rappresenta il gruppo [0]:  
 (('Milano', 'Pressione'), Int64Index([3, 4, 5, 9], dtype='int64'))

 Nome   del gruppo 0 --> items[0][0] : 
 ('Milano', 'Pressione')

 Indici del gruppo 0 --> items[0][1] : 
 Int64Index([3, 4, 5, 9], dtype='int64')

Ottengo le righe dal dataframe in base agli indici del gruppo :

 ----- Riga del dataframe 3
localita       Milano
sensore     Pressione
valore            998
Name: 3, dtype: object

 ----- Riga del dataframe 4
localita       Milano
sensore     Pressione
valore            997
Name: 4, dtype: object

 ----- Riga del dataframe 5
localita       Milano
sensore     Pressione
valore            999
Name: 5, dtype: object

 ----- Riga del dataframe 9
localita       Milano
sensore     Pressione
valore            998
Name: 9, dtype: object

    \end{Verbatim}

    \paragraph{Altro modo per ottenere l'elenco dei
gruppi}\label{altro-modo-per-ottenere-lelenco-dei-gruppi}

    \begin{Verbatim}[commandchars=\\\{\}]
{\color{incolor}In [{\color{incolor}100}]:} \PY{n+nb}{list}\PY{p}{(}\PY{n}{gb2}\PY{o}{.}\PY{n}{groups}\PY{p}{)}
\end{Verbatim}


\begin{Verbatim}[commandchars=\\\{\}]
{\color{outcolor}Out[{\color{outcolor}100}]:} [('Milano', 'Pressione'),
           ('Milano', 'Temperatura'),
           ('Torino', 'Pressione'),
           ('Torino', 'Temperatura')]
\end{Verbatim}
            
    \paragraph{Otteniamo un dataframe in base al gruppo che ci interessa
:}\label{otteniamo-un-dataframe-in-base-al-gruppo-che-ci-interessa}

    \begin{Verbatim}[commandchars=\\\{\}]
{\color{incolor}In [{\color{incolor}108}]:} \PY{n+nb}{type}\PY{p}{(}\PY{n}{gb2}\PY{o}{.}\PY{n}{get\PYZus{}group}\PY{p}{(}\PY{p}{(}\PY{l+s+s1}{\PYZsq{}}\PY{l+s+s1}{Milano}\PY{l+s+s1}{\PYZsq{}}\PY{p}{,}\PY{l+s+s1}{\PYZsq{}}\PY{l+s+s1}{Pressione}\PY{l+s+s1}{\PYZsq{}}\PY{p}{)}\PY{p}{)}\PY{p}{)}
          \PY{n}{gb2}\PY{o}{.}\PY{n}{get\PYZus{}group}\PY{p}{(}\PY{p}{(}\PY{l+s+s1}{\PYZsq{}}\PY{l+s+s1}{Milano}\PY{l+s+s1}{\PYZsq{}}\PY{p}{,}\PY{l+s+s1}{\PYZsq{}}\PY{l+s+s1}{Pressione}\PY{l+s+s1}{\PYZsq{}}\PY{p}{)}\PY{p}{)}
\end{Verbatim}


\begin{Verbatim}[commandchars=\\\{\}]
{\color{outcolor}Out[{\color{outcolor}108}]:}   localita    sensore  valore
          3   Milano  Pressione     998
          4   Milano  Pressione     997
          5   Milano  Pressione     999
          9   Milano  Pressione     998
\end{Verbatim}
            
    \begin{Verbatim}[commandchars=\\\{\}]
{\color{incolor}In [{\color{incolor}1}]:} \PY{k+kn}{from} \PY{n+nn}{IPython}\PY{n+nn}{.}\PY{n+nn}{core}\PY{n+nn}{.}\PY{n+nn}{display} \PY{k}{import} \PY{n}{display}\PY{p}{,} \PY{n}{HTML}
        \PY{n}{display}\PY{p}{(}\PY{n}{HTML}\PY{p}{(}\PY{l+s+s2}{\PYZdq{}}\PY{l+s+s2}{\PYZlt{}style\PYZgt{}.container }\PY{l+s+s2}{\PYZob{}}\PY{l+s+s2}{ width:800px !important; \PYZcb{}\PYZlt{}/style\PYZgt{}}\PY{l+s+s2}{\PYZdq{}}\PY{p}{)}\PY{p}{)}
\end{Verbatim}


    
    \begin{verbatim}
<IPython.core.display.HTML object>
    \end{verbatim}

    

    % Add a bibliography block to the postdoc
    
    
    
    \end{document}
